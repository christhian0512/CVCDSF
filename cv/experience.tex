%-------------------------------------------------------------------------------
%	SECTION TITLE
%-------------------------------------------------------------------------------
\cvsection{Experience}


%-------------------------------------------------------------------------------
%	CONTENT
%-------------------------------------------------------------------------------
\begin{cventries}

%---------------------------------------------------------
  \cventry
    {Applications engineer} % Job title
    {Carvajal Tecnología y Servicios} % Organization
    {Cali, Colombia} % Location
    {Oct. 2016 - PRESENT} % Date(s)
    {
      \begin{cvitems} % Description(s) of tasks/responsibilities
        \item {Implemented a purchasing BI model for Carvajal Educacion Colombia}
        \item {Implemented a SLA BI model for Carvajal Servicios}
        \item {Mantained and improved php applications}
        \item {Worked with interdisciplinary teams over the course of several projects succesfully}
      \end{cvitems}
    }

%---------------------------------------------------------
  \cventry
    {Software Engineer} % Job title
    {Universidad del Valle} % Organization
    {Cali, Colombia} % Location
    {Feb. 2016 -  Jun. 2016} % Date(s)
    {
      \begin{cvitems} % Description(s) of tasks/responsibilities
        \item {Development of the web aplicattion Dicasa to support the data retrieval and analisyss of the existent capabilities regarding enviromental health in the 32 departments and districts of colombia, tasked with supporting the proyect, designing and developing the solution for the data capture}
      \end{cvitems}
    }

  \cventry
    {Student} % Job title
    {Undergraduate Thesis project (Universidad del Valle, Prof. Victor Bucheli, Prof. Beatriz Florian) } % Organization
    {Cali, Colombia} % Location
    {Aug. 2014 -  May. 2015} % Date(s)
    {
      \begin{cvitems} % Description(s) of tasks/responsibilities
        \item {Implemented the prototype mobile application, VT-Mobile, which supports technological vigilance in the Universidad Del Valle. This prototype produces a set of visualizations and indicators from information extracted from a group of Scopus data. To create said visualizations and indicators the data goes through an enrichment process following the activities of: search, extraction, clustering, data analysis and data visualization. This tool makes possible to analyze the scientific production of the Escuela de Ingenieria de Sistemas y Computación.}
      \end{cvitems}
    }


%---------------------------------------------------------
\end{cventries}
